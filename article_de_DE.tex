%% !TEX encoding = UTF-8 Unicode

\documentclass[12pt]{article}

\usepackage[a4paper, left=2cm, right=2cm, top=2cm]{geometry}
\usepackage[utf8]{inputenc}
\usepackage[onehalfspacing]{setspace} %1.5px
\usepackage[ngerman]{babel}
\usepackage{graphicx} % include graphs
\usepackage{mathptmx} %Times New Roman
\usepackage{apacite} %APA 6
\usepackage[hyphens]{url}
\usepackage{acronym} %Abbreviations
\usepackage{tocstyle}  % For points in contents
\usepackage{epigraph}
\usepackage{pdfpages}

\newcommand\tab[1][0.5cm]{\hspace*{#1}}
\newcommand*\wildcard[2][5cm]{\vspace*{2cm}\parbox{#1}{\hrulefill\par#2}}
\newcommand*{\quelle}{% 
  \footnotesize Quelle: 
} 

%Tabellen Abstand der Reihen
\renewcommand{\arraystretch}{1.5}

\begin{document}

% Deckblatt
% Titel Seite
	\begin{titlepage}

		\begin{center}
		\line(1,0){300} \\
		[2mm]
		\huge{\bfseries Artikel} \\
		\line(1,0){300} \\
		[1cm]

		\large{im Kurs \\ Musterkurs} \\
		[1cm]
		\LARGE{\textbf{Titel des Artikels / Arbeit}} \\
		[1.5cm]

		\large{\textbf {vorgelegt von} \\ Name \\ 
		Hochschule \\
		E-Mail Adresse} \\
		[0.5cm]
		\large{\textbf{MatrikelNr.} \\ *******} \\
		[0.5cm]
		\large{\textbf{vorgelegt am} \\ Datum} \\
		[0.5cm]
		\large{\textbf{Prüfer} \\ Name} \\ 

	\end{center}
\end{titlepage}
% Ende Titel Seite

% Inhaltsverzeichnis
\newpage
\newtocstyle[KOMAlike][leaders]{alldotted}{} %Gepunktete Linien
\usetocstyle{alldotted} % Gepunktete Linien
\tableofcontents
% Ende Inhaltsverzeichnis

% Abkürzungsverzeichnis
\newpage
\section*{Abkürzungsverzeichnis}
\addcontentsline{toc}{section}{Abkürzungsverzeichnis} %Section ohne Nummer im Inhaltsverzeichnis
\begin{acronym}[WWW]	%The largest abbreviation 
	\setlength{\itemsep}{-\parsep} % kein Abstand, kompakte Darstellung 
	
	\acro{www}[WWW]{World Wide Web} %comand to create a abbr.

\end{acronym}

\newpage
% List of Figures
\addcontentsline{toc}{section}{Abbildungsverzeichnis}
\listoffigures

% List of Tables
\newpage
\addcontentsline{toc}{section}{Tabellenverzeichnis}
\listoftables
\newpage

% Einleitung
% Einleitung
\newpage
\section{Einleitung}

% Motivation //
\subsection{Motivation} \label{ssec:motivation}

\subsection{Ziele}

\subsection{Aufbau}


\subsection{Methodik} \label{ssec:methodik}




%Grundlagen
% Grundlagen digitaler Sprachassistenten 
\newpage
\section{Grundlagen} \label{sec:grundlagen}

Das ist eine Abbildung: 
\begin{figure}[h]
	\centering
	\includegraphics[width=0.5\textwidth]{Grafiken/hhulogo.png}
	\caption{Logo der Heinrich-Heine-Universität Düsseldorf}
	\quelle (https://www.uni-duesseldorf.de/home/startseite.html)
	\label{img:hhulogo}
\end{figure}

Das ist eine Tabelle: 
\begin{table}[h]
\centering
\begin{tabular}{l*{6}{c}r}
Spalte  & Spalte & Spalte \\
\hline
Wert & Wert & Wert  \\
Wert  & Wert & Wert 
\end{tabular}
\caption{Eine Tabelle in Latex}
\quelle Eigene Darstellung
\label{table:tabelle}
\end{table}

%Empirische Studie
% Empirische Studie 
\newpage
\section{Empirische Studie} 

%Diskussion
% Finale Diskussion
\newpage
\section{Diskussion} \label{sec:disc}


% Literaturverzeichnis
\newpage
\bibliographystyle{apacite} % Citations in APA 6
\bibliography{Literature}

% Include PDF files 
\section{Anhang}
%\includepdf[pages=-]{file.pdf}


\end{document}
